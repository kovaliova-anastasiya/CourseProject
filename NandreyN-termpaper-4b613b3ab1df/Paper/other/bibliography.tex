\newpage
\begin{thebibliography}{3}
    \addcontentsline{toc}{section}{Список использованных источников}

	\bibitem{tanenbaum} Таненбаум Э., Современные операционные системы. 3-е изд., СПб, 2010г., ISBN 978-5-49807-306-4 [Тип ресурса?]/ --Режим доступа: ? -- Дата доступа: 28.11.2018 %marked

	\bibitem{vahalia} Вахалия Ю., Unix изнутри, СПб, 2003г., ISBN 5-94723-013-5 -- Дата доступа: 06.12.2018

	\bibitem{lav} Лав Р., Linux. Системное программирование. 2-е изд., СПб, 2014г., ISBN 978-5-496-00747-4  -- Дата доступа: 04.12.2018
	
	\bibitem{rihter} Рихтер Дж., Windows для профессионалов: создание эффективных Win32 приложений с учетом специфики 64-разрядной версии Windows, 4-е изд., СПб, 2001г., ISBN 5-272-00384-5 -- Дата доступа: 06.12.2018

	\bibitem{hart} Харт, Джонсон, М., Системное программирование в среде Windows, 3-е издание, М., 2005г., ISBN 5-8459-0879-5 -- Дата доступа: 06.12.2018 %marked
	
	\bibitem{nazar} Назар К., Рихтер Дж., Windows via C/C++. Программирование на языке Visual C++, СПб, 2009г., ISBN 978-5-7502-0367-3 -- Дата доступа: 07.12.2018  

	\bibitem{ubuntuman} Ubuntu Manpage: sched [Электронный ресурс]/ --Режим доступа: http://manpages.ubuntu.com/manpages/cosmic/man7/sched.7.html -- Дата доступа: 04.12.2018
	
	\bibitem{ibmLinuxCFS} Inside the Linux 2.6 Completely Fair Scheduler [Электронный ресурс]/ --Режим доступа:  https://developer.ibm.com/tutorials/l-completely-fair-scheduler/ -- Дата доступа: 10.12.2018
	
	\bibitem{ibmLinux} Планировщик задач Linux [Электронный ресурс]/ --Режим доступа:  https://www.ibm.com/developerworks/ru/library/l-scheduler/index.html -- Дата доступа: 02.12.2018
	
	\bibitem{rt} Silberschatz et al., Applied Operating System Concepts, 9th ed, 2013. [Электронный ресурс]/ --Режим доступа:  http://iips.icci.edu.iq/images/exam/Abraham-Silberschatz-Operating-System-Concepts---9th2012.12.pdf -- Дата доступа: 30.11.2018
	\end{thebibliography}

