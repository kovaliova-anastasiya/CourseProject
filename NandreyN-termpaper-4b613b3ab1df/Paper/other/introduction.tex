\newpage

\section*{ВВЕДЕНИЕ}
\addcontentsline{toc}{section}{Введение}
Современные компьютеры выполняют множество задач одновременно. К примеру, во время работы операционной системы одновременно исполняется несколько задач: обновление графического интерфейса, обработка нажатий клавиш клавиатуры, мыши, работа фоновых служб. Для поддержания системы в актуальном состоянии и моделирования параллельной обработки необходима поддержка механизмов и алгоритмов, назначающих задачи на выполнение. В работе будут рассмотрены алгоритмы планирования процессов и их варианты реализаций в некоторых операционных системах. 
 
\subsection*{Основные определения и понятия}
\label{defs}
\addcontentsline{toc}{subsection}{Основные определения и понятия}
\textbf{Блокировка процесса} - остановка выполнения всех инструкций процесса. Одной
из причин появления является необходимость синхронизации выполнения
множества процессов.\\
\textbf{Квант} - временной интервал, в течение которого процесс может выполняться.\\
\textbf{Оборотное время} - среднее время, прошедшее между отправлением задачи на обработку и получением решения. Вычисляется статистически.\\
\textbf{Приоритет процесса} - характеристика процесса, в каком-то смысле обозначает требовательность процесса к ресурсам системы.\\
\textbf{Производительность} - число заданий, выполняемых в единицу времени.\\
\textbf{Процесс} - выполняющаяся программа и все ее элементы: адресное пространство, глобальные переменные, регистры, стек и так далее.\\
\textbf{Система разделения времени} - система, поддерживающая распределение вычислительных ресурсов между многими пользователями с помощью мультипрограммирования и многозадачности.\\
\textbf{Такт процессора} - промежуток времени между двумя последовательными срабатываниями системного таймера.\\
\textbf{ЦП} - центральный процессор.\\ 
является необходимость синхронизации выполнения множества процессов.\\
\textbf{LIL} - List Of Lists - список , каждый элемент которого является списком.\\
Обозначение \textbf{<<} - много меньше.

