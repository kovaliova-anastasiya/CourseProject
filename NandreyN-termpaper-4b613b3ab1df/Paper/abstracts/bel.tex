\begin{center}
\textbf{Рэферат}
\end{center}

Курсавой праект, 34 с., 10 крыніц, 11 табліц.\\

ПАРАЎНАЛЬНЫ АНАЛІЗ КІРАВАННЯ ПРЫКЛАДНЫМІ ПРАГРАМАМІ (ПРАЦЭСАМІ) У АС UNIX-LINUX І WINDOWS.\\

\quad \textbf{Аб'ект даследавання} -- Працэсы. Канцэпцыя працэсу ў сучасных АС. Асаблівасці кіравання ў АС Unix-Linux. Асаблівасці кіравання ў АС Windows.\\

\quad \textbf{Мэта працы} -- Вывучыць асноўныя канцэпцыі і асаблівасці па крыніцах літаратуры. Прааналізаваць і апісаць агульныя рысы і асаблівасці доследнай вобласці ў дадзеных сістэмах. Падрыхтаваць адзін і той жа прыклад праграмы, які дэманструе кіраванне працэсамі, рэалізаваць на мове C/C ++ у кожнай з дадзеных сістэм. Пазначыць вартасці і недахопы кожнага з падыходаў.\\

\quad \textbf{Метады даследавання} -- Праца з літаратурай, падрыхтоўка праграмы.\\

\quad \textbf{Вынікі}  -- Разгледжаны тыповыя падыходы да планавання працэсаў. Разгледжаны падыходы да планавання ў названых АС. Рэалізавана прыкладанне ў кожнай з АС. Прыведзены параўнання часу працы прыкладання ў названых АС.
