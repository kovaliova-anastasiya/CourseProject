ОС FreeBSD и Linux Ubuntu запускались на виртуальной машине Oracle VM VirtualBox в конфигурации : 2 ядра, 1 поток, 2048 МБ оперативной памяти, остальные настройки по умолчанию. Система Windows 10 установлена в качестве основной системы на компьютере. ЦП : Intel Core i7-4500U (2 ядра, 4 потока), 8Гб ОЗУ.\\
Результаты экспериментов представлены в Таблицах \ref{Tables:FreeBSD::tests1}, \ref{Tables:FreeBSD::tests2}, \ref{Tables:Linux::tests2}, \ref{Tables:Linux::tests1}, \ref{Tables:Windows::tests1}, \ref{Tables:Windows::tests2}.

\begin{table}[H]
\begin{tabularx}{\textwidth}{|X|X|X|X|X|X|}
\hline
&Число файлов& 5 & 10 & 20 & 50 \\
\hline
Число строк&&&&&\\
\hline
100 && 15ms & 16ms & 25ms & 50ms\\
\hline
1000 && 40ms & 79ms & 146ms & 433ms\\
\hline
10000 && 412ms & 796ms & 1556ms & 3973ms\\
\hline
100000 && 4196ms & 8500ms & 17804ms & 42490ms\\
\hline
\end{tabularx}
\caption{\textbf{Тесты в ОС FreeBSD при 1 процессе $Sorter$}}
\label{Tables:FreeBSD::tests1}
\end{table}


\begin{table}[H]
\begin{tabularx}{\textwidth}{|X|X|X|X|X|X|}
\hline
&Число файлов& 5 & 10 & 20 & 50 \\
\hline
Число строк&&&&&\\
\hline
100 && 20ms & 23ms & 22ms & 35ms\\
\hline
1000 && 33ms & 60ms & 129ms & 308ms\\
\hline
10000 && 264ms & 602ms & 1191ms & 2975ms\\
\hline
100000 && 2764ms & 5097ms & 10750ms & 25129ms\\
\hline
\end{tabularx}
\caption{\textbf{Тесты в ОС FreeBSD при 2 процессах $Sorter$}}
\label{Tables:FreeBSD::tests2}
\end{table}
% ---------------------------------------------------------------------------

\begin{table}[H]
\begin{tabularx}{\textwidth}{|X|X|X|X|X|X|}
\hline
&Число файлов& 5 & 10 & 20 & 50 \\
\hline
Число строк&&&&&\\
\hline
100 && 6ms & 8ms & 16ms & 32ms\\
\hline
1000 && 27ms & 51ms & 101ms & 212ms\\
\hline
10000 && 264ms & 452ms & 893ms & 2308ms\\
\hline
100000 && 2482ms & 4939ms & 9565ms & 24121ms\\
\hline
\end{tabularx}
\caption{\textbf{Тесты в ОС Ubuntu при 1 процессе $Sorter$}}
\label{Tables:Linux::tests1}
\end{table}


\begin{table}[H]
\begin{tabularx}{\textwidth}{|X|X|X|X|X|X|}
\hline
&Число файлов& 5 & 10 & 20 & 50 \\
\hline
Число строк&&&&&\\
\hline
100 && 4ms & 8ms & 14ms & 24ms\\
\hline
1000 && 17ms & 26ms & 47ms & 143ms\\
\hline
10000 && 222ms & 339ms & 530ms & 1290ms\\
\hline
100000 && 1747ms & 2792ms & 5474ms & 13472ms\\
\hline
\end{tabularx}
\caption{\textbf{Тесты в ОС Ubuntu при 2 процессах $Sorter$}}
\label{Tables:Linux::tests2}
\end{table}
% ---------------------------------------------------------------------------
Поскольку системы FreeBSD и Ubuntu недавно были установлены, то в них не было достаточного количества активных процессов, чтобы протестировать работу программы в условиях конкуренции со стороны остальных активных процессов системы. Как следствие, изменение значения (как увеличение, так и уменьшение) $nice\_value$ для процессов типа $Sorter$ значимых изменений в Таблицы \ref{Tables:FreeBSD::tests1}, \ref{Tables:FreeBSD::tests2}, \ref{Tables:Linux::tests1}, \ref{Tables:Linux::tests2} не внесло. С точки зрения логики, при увеличении числа процессов в $k$ раз ожидается и уменьшение времени выполнения в $k$ раз. Однако это не так. Потеря во времени связана с конвейерной обработкой команд, затратами на работу планировщика процессов, на переключение контекстов процессов и прочими факторами.
Особенно явно эту проблему можно наблюдать при сравнении данных из таблиц \ref{Tables:Windows::tests1}, \ref{Tables:Windows::tests2}.
% ---------------------------------------------------------------------------
\begin{table}[H]
\begin{tabularx}{\textwidth}{|X|X|X|X|X|X|}
\hline
&Число файлов& 5 & 10 & 20 & 50 \\
\hline
Число строк&&&&&\\
\hline
100 && 45ms & 61ms & 74ms & 146ms\\
\hline
1000 && 79ms & 126ms & 222ms & 690ms\\
\hline
10000 && 236ms & 378ms & 871ms & 1897ms\\
\hline
100000 && 927ms & 1720ms & 3784ms & 9914ms\\
\hline
\end{tabularx}
\caption{\textbf{Тесты в ОС Windows 10 при 1 процессе $Sorter$}}
\label{Tables:Windows::tests1}
\end{table} 

\begin{table}[H]
\begin{tabularx}{\textwidth}{|X|X|X|X|X|X|}
\hline
&Число файлов& 5 & 10 & 20 & 50 \\
\hline
Число строк&&&&&\\
\hline
100 && 142ms & 121ms & 150ms & 183ms\\
\hline
1000 && 157ms & 214ms & 299ms & 579ms\\
\hline
10000 && 287ms & 434ms & 1019ms & 1110ms\\
\hline
100000 && 850ms & 1125ms & 2147ms & 5120ms\\
\hline
\end{tabularx}
\caption{\textbf{Тесты в ОС Windows 10 при 4 процессах $Sorter$}}
\label{Tables:Windows::tests2}
\end{table}
Таким образом, при больших объемах и количествах файлов выгоднее использовать обработку несколькими процессами вместо последовательной обработки. Исходя из результатов тестирования, время выполнения в ОС Windows 10 удалось уменьшить примерно в 2 раза при стандартных значениях приоритетов. Были предприняты попытки изменять класс приоритетов процессов в ОС Windows 10, однако время выполнения сократить не удалось. Предлагаю следующее объяснение:\\
''Узким местом'' реализации является чтение текстовых файлов с жесткого диска компьютера. В языке C++ используется буферизированный потоковый ввод, то есть файл читается не целиком, а кэшируется некоторая его часть. В реализации после чтения очередной строки происходит ее добавление в $vector$,после чего считывается следующая строка и так далее, то есть устройство ввода-вывода используется очень интенсивно, даже несмотря на кэширование данных. Как отмечалось выше, для хранения содержимого файлов используется контейнер $vector$, причем данные в него добавляются с помощью операции $push\_back(T)$. Проведем еще одну серию тестов, где вместо $vector$ будет использован $deque$. Результаты отражены в Таблице \ref{Tables:Windows::comp}

\begin{table}[H]
\begin{tabularx}{\textwidth}{|X|X|X|X|X|X|}
\hline
&Число файлов& 5 & 10 & 20 & 50 \\
\hline
Тип контейнера, число процессов&&&&&\\
\hline
vector, 1 Sorter && 927ms & 1720ms & 3784ms & 9914ms\\
\hline
deque, 1 Sorter && 1359ms & 2455ms & 4652ms & 12603ms\\
\hline
vector, 4 Sorters && 850ms & 1125ms & 2147ms & 5120ms\\
\hline
deque, 4 Sorters && 765ms & 1242ms & 2315ms & 5322ms\\
\hline
\end{tabularx}
\caption{\textbf{Тесты в ОС Windows 10 при 4 процессах $Sorter$}}
\label{Tables:Windows::comp}
\end{table}

Полученные результаты косвенно подтверждают, что самая дорогостоящая по времени часть программы - именно ввод-вывод. Возможно, поэтому изменение класса приоритета для процессов типа $Sorter$ существенно не влияют на итоговое время работы программы.