% Таблица - Рихтер, Назар
В Windows начиная с версии NT используется алгоритм, схожий с рассмотренным в Linux и 4.3BSD Unix. Главное различие - в системе приоритетов \cite{rihter} \cite{hart} . Система классов приоритетов \textbf{процессов} описана в Таблице {\ref{Tables:WinPrior}} 

\begin{table}[H]
\begin{tabularx}{\textwidth}{|X|X|X|}
\hline
\textbf{Название класса приоритетов} & \textbf{Краткое описание}\\
\hline
Idle & Процесс выполняется, когда система не занята другой работой. Подходит для фоновой работы\\
\hline
Below normal & Промежуточный класс\\
\hline
Normal & Основной класс приоритетов, который используют большинство приложений\\
\hline
Above normal & Промежуточный класс\\
\hline
High & Процесс с приоритетами из этого класса обязан немедленно реагировать на события\\
\hline
Real-time & Приоритеты процессов реального времени, такие процессы имеют право вытеснять даже компоненты ОС\\
\hline
\end{tabularx}
\caption{\textbf{Классы приоритетов процессов в Windows}}
\label{Tables:WinPrior}
\end{table}

С классом приоритетов процесса связаны приоритеты его потоков. После присвоения процессу класса приоритетов, необходимо указать относительные приоритеты потоков (Таблица \ref{Tables:ThRelativePr} ) в пределах процесса, которому поток принадлежит. Числовое значение приоритета формируется самой системой исходя из класса приоритетов процесса и относительного приоритета потока. Числовые значения часто меняются в процессе развития и доработки системы.

\begin{table}[H]
\begin{tabularx}{\textwidth}{|X|X|X|}
\hline
\textbf{Относительный приоритет потока} & \textbf{Краткое описание}\\
\hline
Idle & Приоритет потока равен 16 в классе real-time и 1 в остальных классах\\
\hline
Lowest; Below normal; Normal; Above Normal&К обычному приоритету для класса прибавляется значение с соответствующим номером из списка [-2; -1; 0; 1; 2] \\
\hline
Time-critical& Приоритет потока равен 31 в классе real-time и 15 в других процессах\\
\hline
\end{tabularx}
\caption{\textbf{Относительные приоритеты потоков в Windows}}
\label{Tables:ThRelativePr}
\end{table}
