Для каждой вершины случайным образом равновероятно выбираем вершину из списка смежности, в которую попытаемся отправить информацию, только теперь сделаем так, чтобы выбранная вершина гарантированно была вершиной, в которой ещё нет информации.

После того, как для какой-то их вершин мы выбрали кандидата, нужно сделать так, чтобы остальные вершины без кандидата не рассматривали кандидатов других вершин, а для этого её нужно удалить из всех списков смежности, где этот кандидат имеется. Если использовать бинарную кучу, из которой можно удалять по ссылке произвольный элемент, или декартово дерево, удаление из списков будет выполняться за логарифм от размеров списка. Использование структуры данных, позволяющей удалять из себя за $\mathcal{O}(1)$, не представляется возможным, так как эта структура должна позволить нам огранизовать равноверятный случайный выбор элемента. 

В результате алгоритм будет работать в худшем случае за $\mathcal{O}(n^3 \log n)$, это если польоваться структурами данных, позволяющим делать более быстрое удаление. Реализованный мной алгоритм в худшем случае работает за $\mathcal{O}(n^2m)$. 

Отметим также, что на разреженных графах время работы не будет соответствовать заявленной асмимптотике, потому что размеры списков будут относительно малы, и удаление соответственно будет выполняться быстрее. А если граф будет плотный, то значение будет близко к $\lceil \log _2 n \rceil$ (утв. 2.1).
