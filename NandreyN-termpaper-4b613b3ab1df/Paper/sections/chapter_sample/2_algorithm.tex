Для вершины, имеющей информацию, из её списка смежности равновероятно случайным образом выбирается вершина, куда пробуем отправить информацию. Если <<оракул>> указал на вершину, в которой нет информации, то тогда эту вершину помечаем как ту, в которую уже не нужно ничего отправлять, и отправляем в буфер из вершин, которые попадут в множество с информацией после окончания шага, в противном случае просим <<оракула>> ещё раз указать на вершину. В качестве параметра передаётся константа $c$ -- максимальное число раз, которые можно попросить <<оракула>> указать на вершину на одном шаге.

Этот алгоритм может работать очень долго, если оракул неудачно выбирает, а минимальное число шагов не превышает порядок графа, поэтому алгоритм будет останавливаться, если число шагов достигло значения порядка (утв. 2.2).

Отметим, что для организации случайного выбора вершины-кандидата на отправление для простоты множество вершин, смежных с данной, хранилось в структуре данных с возможностью доступа по индексу за константное время, т. к. оракул будет указывать на индекс в этой структуре.

Время работы – $\mathcal{O}(n^2)$, т. к. каждый выбор кандидата и обработка удачного/неудачного выбора проводится за константное время, а всего таких выборов на каждом шаге линейное число, и число шагов тоже линейно. 
