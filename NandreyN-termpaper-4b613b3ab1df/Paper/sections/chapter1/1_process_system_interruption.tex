% Источник : Операционная система UNIX
% Робачевский Андрей М.
Компьютер всегда имеет встроенный таймер, срабатывающий через определенные временные интервалы. Например, стандартное время между 2 последовательными срабатываниями (такт процессора) для ОС Windows - около 16 мс, для Unix - около 10мс. Само по себе окончание такта процессора ничего не значит, но оно инициирует прерывание таймера.

\label{interruption:timer}Некоторые из задач, выполняемые обработчиком прерывания таймера:
\begin{enumerate}[label=---]
\item Действия, связанные с планированием процессов.
\item Происходит пересчет времени, в течение которого процесс занимал процессор.
\item Обновление системного времени.
\end{enumerate}
Из-за того, что таймер срабатывает достаточно часто, обработка таких прерываний является одной из наиболее важных задач в системе управления процессами.\\

