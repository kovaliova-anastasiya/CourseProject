В интерактивных системах можно воспользоваться идеями алгоритма SJF [см. \ref{package:sjf}], поскольку достоинства SJF вписываются в их концепцию. В алгоритме SJN приоритетом будет предположительное время выполнения процесса, для вычисления которого рассмотрим несколько подходов:

\begin{enumerate}[label=---]
\item \textbf{Статическое предсказание.} Основывается на информации о процессах, которая, как следует из названия, является статической, то есть не изменяется во время выполнения процесса. Предположения о продолжительности выполнения можно выдвигать на основании размера дискового пространства, занимаемого исходным кодом программы, выполняющейся в процессе; на основании типа процесса в том смысле, что утилита командной строки, скорее всего, выполнится быстрее другого пользовательского процесса. Однако, такие оценки  часто могут оказываться неверными и малоэффективными.

\item \textbf{Динамическое предсказание.} Основывается на информации о процессе, полученной непосредственно во время выполнения. Предположим, что имеется информация о времени работы $T_1, T_2, ..., T_n$ конкретного процесса $P$ исходя из предыдущих запусков этого процесса. На основании этих данных вводится \textbf{оценка по сроку давности (aging)}: $T = \sum_{k=0}^{n-1}{(1 - \alpha)^{k}T_{n-k}} $. Параметр $a \in [0,1]$ позволяет регулировать вклад запусков в оценку времени работы. Часто бывает удобно принимать $a = \frac{1}{2}$ в силу простоты деления на степень двойки.
\end{enumerate}

Достоинства:
\begin{enumerate}[label=---]
\item Учет оценки времени выполнения при планировании.
\item Пригоден для использования в системах, где работа пользователя завязана на системных утилитах (позволяет задействовать оценку по сроку давности).
\end{enumerate}

Недостатки:
\begin{enumerate}[label=---]
\item Высокое время оборота для требовательных к системным ресурсам процессов.
\item Внесение изменений в исходный код программы разработчиком, вообще говоря, сводит на нет усилия по формированию оценки по сроку давности (Может быть добавлен вызов подпрограммы, выполнение которой требует большого процессорного времени).

\item Информацию, необходимую для оценивания по сроку давности, нужно хранить таким образом, чтобы скорость доступа к ней была крайне высока, поскольку от этого напрямую зависит быстродействие всей системы. 
\end{enumerate}