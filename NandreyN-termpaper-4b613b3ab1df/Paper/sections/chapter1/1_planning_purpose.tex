Рассмотрим компьютер с одноядерным процессором и одним процессом $A$, исполняющим, например, алгоритм решения задачи Коши для дифференциального уравнения в частных производных. Решение задачи может занимать длительное время в силу того, что решение вычисляется на двумерной сетке узлов. Пока процесс в системе единственен, проблем не возникает. \\ Пусть появилась возможность запустить еще один процесс $B$, единственная цель которого - вывести на экран результат операции "$2 * 6$". Если не вмешиваться в выполнение  процесса $A$, то он будет выполняться до тех пор, пока не выполнит всю необходимую работу, затем он \textbf{добровольно} освободит ЦП, а его место займет процесс $B$. Однако, не совсем правильно, что процесс с малым потреблением ресурсов $B$  должен дожидаться окончания выполнения требовательного к ресурсам процесса $A$. Более того, процессы могут использовать устройства ввода-вывода, сетевое соединение, взаимодействовать между собой и другими объектами системы, что представляет дополнительные сложности при планировании выполнения процессов.\\

Для разрешения вопроса, какой процесс должен вскоре выполняться, вводятся \textbf{алгоритмы планирования}, а исполняющий их объект называется \textbf{планировщиком процессов}.
\\
Для некоторых платформ критично потребление ими электроэнергии, поэтому перед планировщиком процессов может ставиться задача повышения энергоэффективности:\\
Пусть имеется $n$ процессов $P_1, P_2, ... , P_n$. Необходимо выработать такую стратегию назначения процессов на выполнение, чтобы минимизировать время простоя процессора.\\
В описанных выше ситуациях считалось, что все процессы не имеют абсолютно никакого отношения друг и другу, т.е. являются полностью независимыми. При решении некоторых задач бывает возможно сформулировать алгоритм решения так, чтобы какие-то части могли выполняться параллельно и независимо друг от друга. Например, при выполнении умножения матрицы на вектор происходит умножение каждой отдельной строки на этот вектор, независимо друг от друга. Такие операции при больших размерностях часто бывает выгодно проводить в отдельном процессе (при условии, что выигрыш во времени вычислений покроет издержки на создание процессов и затраты на планирование ), аккумулируя результат позже.
