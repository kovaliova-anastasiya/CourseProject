Свяжем приоритеты $pr_i(t)$ процессов $p_i$ в текущий момент времени $t$ с расстоянием до крайних сроков их завершения $T_i$, $i = 1..n$, например, следующим образом : $pr_i(t) = \frac{1}{T_i - t}$. Суть алгоритма заключена в следующем:\\
\begin{enumerate}[label=---]
\item Если в системе нет активных процессов реального времени, и поступает сигнал на инициализацию нового процесса, новый процесс создается и назначается на выполнение, вытесняя текущий процесс.
\item Пусть в системе есть активные процессы реального времени. При создании нового процесса порождается системное прерывание, обработчик которого должен выполнить перерасчет приоритетов, так как может оказаться, что вновь поступивший процесс должен быть завершен раньше, чем все остальные процессы. На выполнение будет назначен только что поступивший процесс, поскольку его приоритет самый высокий в силу введенной формулы.
\item При завершении выполнения некоторого процесса реального времени также выполняется перерасчет приоритетов, на выполнение назначается процесс с наибольшим приоритетом.
\end{enumerate}

Описание алгоритма объясняет его название \cite{rt}: EDF - Earliest-Deadline-First Scheduling - Алгоритм планирования по ближайшему сроку завершения.\\
Стоит отметить, что приоритеты процессов являются динамическими, то есть могут и будут изменяться в процессе работы планировщика, причем значение приоритета обратно пропорционально времени нахождения процесса в очереди на выполнение. Более того, алгоритм EDF накладывает существенное ограничение: планировщику должны быть предоставлены крайние сроки завершения обработки процессов от каждого внешнего детектора (их нужно каким-то образом рассчитывать). Утверждается, что алгоритм является теоретически оптимальным в том смысле, что если существует такое распределение процессов в очереди на выполнение, что каждый будет завершен до своего крайнего срока, то оно будет найдено алгоритмом. Однако, на практике с этим могут возникнуть проблемы из-за издержек на промежуточные работы по смене контекста процесса. В этом случае крайние сроки придется сместить.