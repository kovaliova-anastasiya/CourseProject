\label{package:sjf}
Пусть имеется один процессор, зафиксировано множество процессов $\{p_i\}, i = 1..n$ и заранее известно необходимое время выполнения каждого процесса $\{T_i\}, i =1..n$. Ставится задача минимизация среднего оборотного времени процесса. Рассмотрим алгоритм ее решения:
\begin{enumerate}[label=---]
\item Отсортируем процессы таким образом, что $T_{k_1} \leq T_{k_2} \leq ... \leq T_{k_n}$ и поместим в стек (самый быстрый процесс на верхушке).
\item Пока стек не пуст, после завершения очередного процесса извлекаем новый процесс из стека и назначаем на исполнение.
\end{enumerate}
Утверждается, что при таком назначении среднее время оборота процессов будет минимальным. Покажем это.\\
Пусть значения времени выполнения процессов расположены в стеке в произвольном порядке. Тогда для первого процесса время оборота будет $T_{k_1}$, для второго в стеке $T_{k_1} + T_{k_2}$, для $j$-го процесса : $\sum_{i=1}^j{T_{K_i}}$. Среднее время оборота равно среднему арифметическому: $T = \frac{\sum_{l = 1}^n{\sum_{i = 1}^{j}{T_{k_i}}}}{n} \rightarrow min$. Расписав сумму, нетрудно получить, что $T = T_{k_1} + \frac{n-1}{n}T_{k_2} + \frac{n-2}{n}T_{k_3} + ... + \frac{1}{n}T_{k_n}$. Понятно, что взаимный порядок $T_{k_i}$ для минимизации $T$ нужно выбрать следующим: $T_{k_1} \leq T_{k_2} \leq ... \leq T_{k_n}$.\\
Приведенный выше алгоритм известен под названием "Сначала самое короткое задание" - Shortest job first - (SJF) \cite{tanenbaum}.\\
Недостатки:
\begin{enumerate}[label=---]
\item Требуется, чтобы новые процессы не поступали на протяжении всего времени выполнения.
\item Если разрешить добавлять новые процессы, то поступивший процесс нужно добавить в стек, что в худшем случае требует $O(n)$ времени и $O(n)$ дополнительной памяти.
\item Время работы каждого процесса также должно быть заранее известно. Частично этот недостаток можно нивелировать методами оценки времени выполнения.
\item Предполагается, что процессы не прерываются на работу с вводом-выводом, а занимают непрерывный временной интервал для работы с ЦП.
\end{enumerate}
Достоинства:
\begin{enumerate}[label=---]
\item Позволяет минимизировать среднее оборотное время при выполнении всех ограничений.
\item На базе SJF можно строить дальнейшие модификации.
\end{enumerate}
Рассмотрим одну из таких модификаций. Введем приоритет ''оставшееся время выполнения''. Чем меньше это числовое значение, тем выше приоритет. Теперь планировщик будет использовать приоритетную очередь вместо стека. Модификация позволяет быстрее добавлять новые процессы в очередь, а также задачи, не требовательные к ресурсам, будут иметь низкое оборотное время, поскольку будут находиться в начале очереди. С другой стороны, остается не решенной проблема с выполнением требовательных к ресурсам процессов, если в очередь постоянно поступают процессы, которые можно завершить за короткий промежуток времени.
 