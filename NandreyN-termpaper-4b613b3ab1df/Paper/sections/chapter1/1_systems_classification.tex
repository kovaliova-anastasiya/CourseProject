Всякий алгоритм имеет свои слабые и сильные стороны, которые определяют сферу его применения. Алгоритмы планирования выполнения процессов не являются исключением. Стратегия работы планировщика процессов должна 
удовлетворять требованиям к системе, в которой он используется. Различают три основных типа сред выполнения \cite{tanenbaum}:
\begin{enumerate}[label=---]
\item \textbf{Системы реального времени}. Основная задача такой системы - завершение выполнения задачи в пределах жестких временным рамок. Как пример можно привести работу автопилота в современных самолетах. При возникновении непредвиденных факторов во время полета скорость отклика автопилота очень критична, поскольку этим определяется, будет ли сохранен контроль над полетом или будет утрачен.\\ Планировщик в такой системе должен планировать задачи так, чтобы каждая из них уложилась в свои временные рамки, обеспечить необходимое время отклика.
\item \textbf{Системы пакетной обработки}. Основное предназначение - решение задач вычислительного характера, которые не требуют немедленного результата. Главными показателями качества работы системы пакетной обработки являются производительность и оборотное время. Существует третий показатель - загрузка центрального процессора, но он не несет важной информации. Соответственно, алгоритм планирования должен быть выбран таким образом, чтобы значительно увеличить или показатель производительности, или показатель оборотного времени, или оба,но в меньшей степени.
\item \textbf{Интерактивные системы}. Для этого типа систем критическим является минимизация времени отклика на некоторое действие. Например, пользователь, запуская терминал и выполняя в нем команду поиска файлов с определенным шаблоном имени в текущей директории, ожидает начать получать результат практически немедленно, а не после того, как фоновый процесс загрузки обновлений для ОС завершит свою работу. Задачей планировщика в этом случае является обеспечение времени отклика, возможно, в ущерб некоторым другим (фоновым) процессам.
\end{enumerate}
