Можно сказать, что процесс характеризуется своим контекстом. \textbf{Контекст процесса} включается в себя: значения регистров, указатель стека, текущее состояние процесса, счетчик команд, идентификатор, информация об использованном процессорном времени, а для многопроцессорных систем может также указываться конкретный тип процессора, на котором желательно выполнение.\\
Предположим, что необходимо сменить текущий исполняющийся процесс на другой. Каждый из процессов должен работать только со своим контекстом, чтобы обеспечить корректность вычислений. Таким образом , возникает необходимость сменить текущий контекст на нужный. Контекст каждого процесса можно представить как вектор, условно разбитый на блоки. Например, первый блок может содержать id процесса, второй - состояние регистров и так далее. Объединив все такие векторы, получим таблицу, которая называется \textbf{таблицей процессов}. Таким образом, при смене текущего исполняющегося на ЦП процесса операционной системе достаточно обращаться в таблицу процессов, чтобы получить необходимый контекст.