\label{int:cyclical}
Наиболее простым алгоритмом планирования в интерактивных системах является циклический алгоритм \cite{tanenbaum} \cite{ubuntuman} . Вводится временной квант, в течение которого процесс может занимать ЦП. В данной реализации квант связан не с процессом, а с процессором. Планировщик процессов организует FIFO очередь, в которую помещаются все процессы в порядке поступления на выполнение. Предположим, что в данный момент времени ЦП занят некоторым процессом. Возможны следующие ситуации:
\begin{enumerate}[label=---]
\item Временной квант исчерпан, процесс завершен. В этом случае он снимается с выполнения, на его место поступает процесс из очереди.
\item Временной квант исчерпан, но процесс не завершен. Он снимается с выполнения и помещается в конец очереди, его место занимает процесс из очереди.
\item Временной квант не исчерпан, но процесс завершил выполнение или перешел в состояние блокировки. На исполнение поставим новый процесс из очереди, а текущий либо перенесем в конец, либо завершим.
\end{enumerate}
Замечание. Квант обновляется, когда происходит смена контекста процесса и является одинаковым для всех процессов в очереди.\\

Достоинства:
\begin{enumerate}[label=---]
\item Простота реализации. Для хранения используется очередь, а такой тип квантования можно реализовать с помощью механизма прерываний, в частности, с помощью прерываний системного таймера [см. \ref{interruption:timer}]. Стоит учесть, что смена контекста, вообще говоря, может происходить не на каждом прерывании таймера.
\item Не дает требовательным к ресурсам процессам надолго занимать ЦП.
\end{enumerate}

Недостатки:
\begin{enumerate}[label=---]
\item Не подходит для процессов, активно работающих с внешними устройствами. При ожидании результата операции, как уже отмечалось, процесс переходит в конец очереди и вынужден ожидать дольше, чем мог бы.
\item Сложность выбора длительности кванта времени. Длительность тесно связана с эффективностью реализации смены контекста. Если квант времени сделать слишком малым, то расходы на смену контекста могут занимать значительную долю процессорного времени. Если квант окажется слишком большим, то такое планирование перестанет удовлетворять требованиям интерактивных систем - пользователь будет ожидать результата команды непозволительно долго.
\end{enumerate}

Рассмотрим модификацию циклической схемы. Временной квант теперь будет ассоциирован с процессом, а не процессором. Числовая характеристика процесса будет храниться в его записи в таблице процессов. Заводится две очереди. В одной из них хранятся процессы, которые не исчерпали свой квант, в другой - процессы, полностью исчерпавшие свой квант.
При смене контекста процесса будет приниматься решение, в какую из очередей поместить процесс, снимающийся с выполнения (исходя из того, исчерпал ли он свой квант). При добавлении процесса в очередь с процессами, исчерпавшими свой квант, обновим его личный квант. Если очередь с не исчерпавшими квант процессами опустеет, инвертируем роли очередей.