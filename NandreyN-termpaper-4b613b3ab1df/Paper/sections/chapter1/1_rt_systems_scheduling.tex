% Silberschatz et al., Applied Operating System Concepts, 9th ed.
Многие системы реального времени используют механизм событий для функционирования. Например, есть некоторое количество физических устройств - детекторов, которые при фиксировании заданного события отправляют запрос компьютеру, сообщающий, что этот сигнал нужно обработать, причем в сжатые сроки. Различают системы реального времени с \textbf{мягкими} и \textbf{жесткими} ограничениями.\\
Системы реального времени с \textbf{мягкими} ограничениями не дают никакой гарантии относительно крайнего срока завершения процесса, но просто гарантируют, что такому процессу будет оказано предпочтение при планировании. Системы реального времени с \textbf{жесткими} ограничениями, напротив, дают жесткую гарантию относительно времени завершения процесса. Стоит заметить, что реализация системы реального времени требует не только реализации специального алгоритма планирования, но также изменений в системе в целом. Так, например, если рассматривать сигналы от внешних детекторов как прерывания, то необходимо серьезное усовершенствование системы смены контекста процесса. При возникновении прерывания нужно сохранить контекст текущего процесса, загрузить обработчик прерывания,выполнить обработку, вернуть контекст прерванного процесса на исполнение.\\

